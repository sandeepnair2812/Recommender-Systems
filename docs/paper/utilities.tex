\subsection{Recommenders Utilities}

{\em Utilities} are functions designed as the minimum layer needed for supporting common recommendation pipelines.
Some utility functions are designed to support atomic operations, e.g., the utilities invoking the Surprise \cite{Surprise} or 
FastAI \cite{howard2018fastai} libraries, while some others are meant for full implementation of fully functional component, e.g., a recommender algorithm. Utiliy functions can be generally categorized as follows
%NOTE: instead of citing the exact names of the folders that can change in the future, I list the high level concepts
\begin{itemize}
    \item \textbf{Common utilities:} supporting utilities like timers, loggers, GPU functions, 
    Spark helpers, memory management etc.
    \item \textbf{Data preparation:} Utility functions for data download, loading, 
    transformation, splitting etc., which are frequent tasks in recommendation systems development. 
    \item \textbf{Algorithms:} algorithms implementations and/or their auxiliary functions, e.g., xDeepFM \cite{lian2018xdeepfm}, DKN \cite{wang2018dkn}, SAR \cite{diev2015sar} and RLRMC \cite{rlrmc} from Microsoft, NCF 
    \cite{he2017neural}, Wide and Deep \cite{cheng2016wide}, and RBM \cite{salakhutdinov2007restricted} from community. 
    Some well-known Microsoft libraries like LightGBM \cite{ke2017lightgbm} or Vowpal Wabbit \cite{agarwal2014reliable} are also wrapped as utilities.
    \item \textbf{Evaluation:} rating and ranking metrics implemented in Python+CPU and PySpark environments.
    \item \textbf{Model Selection and Optimization:} utility functions developed on top of the NNI hyperparameter 
    tuning framework \cite{nni}. 
    \item \textbf{Operationalization:} model operationalization functions on platforms like Kubernetes.
\end{itemize}
    

