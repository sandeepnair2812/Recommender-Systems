\documentclass{article}

% if you need to pass options to natbib, use, e.g.:
%     \PassOptionsToPackage{numbers, compress}{natbib}
% before loading neurips_2019

% ready for submission
% \usepackage{neurips_2019}

% to compile a preprint version, e.g., for submission to arXiv, add add the
% [preprint] option:
%     \usepackage[preprint]{neurips_2019}

% to compile a camera-ready version, add the [final] option, e.g.:
     \usepackage[final]{neurips_2019}

% to avoid loading the natbib package, add option nonatbib:
%     \usepackage[nonatbib]{neurips_2019}

\usepackage[utf8]{inputenc} % allow utf-8 input
\usepackage[T1]{fontenc}    % use 8-bit T1 fonts
\usepackage{hyperref}       % hyperlinks
\usepackage{url}            % simple URL typesetting
\usepackage{booktabs}       % professional-quality tables
\usepackage{amsfonts}       % blackboard math symbols
\usepackage{nicefrac}       % compact symbols for 1/2, etc.
\usepackage{microtype}      % microtypography

\title{Microsoft-Recommenders: Best Practices for Production-Ready Recommendation Systems}

% The \author macro works with any number of authors. There are two commands
% used to separate the names and addresses of multiple authors: \And and \AND.
%
% Using \And between authors leaves it to LaTeX to determine where to break the
% lines. Using \AND forces a line break at that point. So, if LaTeX puts 3 of 4
% authors names on the first line, and the last on the second line, try using
% \AND instead of \And before the third author name.

\author{%
  Name \\
  Microsoft, Azure Global \\
  2 Kingdom St, London W2 6BD, UK \\
  \texttt{name@microsoft.com} \\
  % examples of more authors
  % \And
  % Coauthor \\
  % Affiliation \\
  % Address \\
  % \texttt{email} \\
  % \AND
  % Coauthor \\
  % Affiliation \\
  % Address \\
  % \texttt{email} \\
  % \And
  % Coauthor \\
  % Affiliation \\
  % Address \\
  % \texttt{email} \\
  % \And
  % Coauthor \\
  % Affiliation \\
  % Address \\
  % \texttt{email} \\
}

\begin{document}

\maketitle

\begin{abstract}
Recommendation systems have become one of the most widely applied areas of machine learning in modern business and 
have attracted great interest in machine learning research. Recommendation algorithms are involved in a large percentage of 
consumer purchases in today's e-commerce production systems. We present {\em Microsoft-Recommenders},  
an open-source Github repository for helping researchers, developers and non-experts in general to select, prototype, train and
bring to production a variety of state-of-the-art recommendation algorithms.
A focus of this repository is on {\em best practices} in development of recommendation systems and Python software for machine learning.
We have also incorporated learnings from our experience with recommendation systems in production, in order to enhance ease of use; speed of 
implementation and deployment; ease of evaluation; scalability and performance. 
The repository has also received contributions from the broader community of researchers and data science practitioners. 
In turn, the repository is a useful aid to anyone in this community and simplifies development and application of recommendation algorithms.
\end{abstract}

\section{Introduction}

\section{Introduction}

Recommendation systems have become ubiquitous in modern business and have attracted a great deal of interest in academic research.
According to McKinsey \& Co,
“35\% of what consumers purchase on Amazon and 75\% of what they watch on Netflix come from recommendations algorithms" 
\cite{mckinsey}.
Recommendations appear almost everywhere in today's e-commerce, in which consumers typically have a large variety of products or other
options to choose from. Examples include recommendation of brands, news, consumer products, media content, 
travel packages etc. In such scenarios the goal of business is usually to increase revenue,
customer engagement and satisfaction, induce networking effects etc. Another related goal may be to improve prediction of the metrics of interest, 
so that better business decisions about advertising spend and internet traffic can be made. In addition, vendors want to better understand their customers,
or segments of customers, in order to address them with more personalized and suitable marketing campaigns.

The history of recommendation algorithms goes back to the 1990s \cite{Tapestry, grouplens}
when the first {\em content-based and collaborative filtering} methods appeared.
Around 2006, the Netflix prize \cite{netflix} provided an addiitional boost to research in the area, in particular leading to significant advances in factorization-based models \cite{bell_lessons,koren2009matrix,SVD++,PMF}.
Later on, more {\em hybrid methods} incorporating standard supervised learning into recommendations as well as more hybrid factorization models have appeared
\cite{rendle,ffm,bpr,pairwise,multiverse}.
In more recent years, {\em deep learning} methods, after their success in other areas of machine learning, have been applied increasingly in the area of recommendations
\cite{PNN,cheng2016wide,lian2018xdeepfm,he2017neural,youtube,nvidia,survival}.
Moreover, the objectives in recommendation problems are broader than, say, prediction accuracy or precision of recommendations, and include a diverse set of tasks and goals such as 
{\em explainable recommendation} \cite{explainable,rl_explainable}, corrections for {\em biases} and offline {\em evaluation} \cite{mnar,offline,debiasing,counterfactual},
{\em reinforcement learning} \cite{rl_explainable,rl,rl_negative} etc.
Recommendation problems are also connected to other areas in machine learning, such as multitask learning \cite{caruana,baxter,mtl,evgeniou}, transfer learning \cite{raina,thrun,boost_transfer} 
and signal processing and statistics \cite{tao,recht,tsybakov}.

In practice, however, application of recommendation algorithms has encountered significant challenges. 
First, there are limited references and guidance for building recommendation systems at scale to support
enterprise grade scenarios. In addition, even though there exist several off-the-shelf packages, tools and modules, they tend to focus on specific aspects of recommendations 
and may not be compatible with each other. Whereas new algorithms are being published continuously in research, in the field many practitioners may lack the 
expertise or time to implement and deploy these algorithms in production. On the other side, researchers, when applying their methods to real world scenarios, may lack experience of the domain of interest
or awareness of the best practices with respect to data science and software engineering. Thus, it can be time-consuming to build a new recommendation system from an
algorithm, even when avaialble as software, or to incorporate it into an existing recommendations pipeline. It can also be time-consuming to build a prototype of a new or existing algorithm and to 
compare the performance of different algorithms on the same recommendations task. 

In response to these challenges, our team has developed {\em Microsoft-Recommenders}, an open-source 
Github repository available at \url{https://github.com/Microsoft/Recommenders}.
The development of this repository is an ongoing {\em collaborative effort} of machine learning researchers and data scientists from the 
Azure Cloud and Microsoft Research organizations, as well as contributors from the outside community such as academic researchers and data scientists.
Broadly speaking, the repository mainly contains 
\begin{itemize}
\item
Utilities: modular functions for data manipulation, evaluation etc.
\item
Algorithms: functions for model training and prediction / recommendation, including factorization methods, deep learning, hybrid methods etc.
\item
Jupyter notebooks: how to examples for building end-to-end recommenders, hands-on familiarization with the algorithms and quick prototyping. 
\end{itemize}
The development approach we follow benefits from existing best practices and software libraries already avaliable in the recommendations community.
In addition, we have strived to make the repository modular and easy to understand and use, so that researchers and practitioners can use it to build
new or customized algorithms, utilities and notebooks.

In the following sections, we summarize the content of Microsoft-Recommenders and present the approach and the principles we have followed
for its development. In Section 2, we provide an overview of the structure and content of the repository. In Section 3, we explain our development approach, some of
the design decisions and show how practitioners are enabled to use and contribute to the repository productively. In Section 4, we discuss how the recommendations
algorithms can be brought in operation and how to incorporate them within end-to-end pipelines, based on our experience with real-life production systems. 
Finally, in Section 5, we conclude with some insights and learnings we have gained from the practical application of the recommendations repository.



\section{Overall Approach and Technologies}

\section{Overall Approach and Technologies}

The repository is open-source under the MIT License and contributions that follow the guidelines are encouraged.  
The development follows standard Github practices such as issues, milestones and pull requests.

All the algorithms and utilities have been written in the Python programming language and the demostrations are 
in the form of Jupyter notebooks \cite{jupyter}.

The platforms supported are Linux and Windows, on a local computer, on premises, or on the Azure Cloud \cite{azure}, depending on availability of resources.
Some of the algorithms can take advantage of GPU resources if available and some others require a Spark environment \cite{spark}.


\section{Coding Guidelines}

\subsection{Code Design}
\label{code-style}

We strive to maintain high quality code to make the utilities in the repository easy to 
understand, use, and extend. We also work hard to maintain a friendly and constructive 
environment. We've found that having clear expectations on the development process 
and consistent style helps to ensure everyone can contribute and collaborate effectively.

We have published in the repository wiki the coding 
guidelines\footnote{\url{https://github.com/Microsoft/Recommenders/wiki/Coding-Guidelines}\label{foot_code_guidelines}} 
for the project. Next, we describe the most important parts.

\subsubsection{Evidence-Based Software Design}
At its core, Evidence-Based Design (EBD) \cite{kembel2012architectural,joeglekar2018evidence} embodies the 
scientific method: it empowers one to ask the right questions and develop hypotheses, 
gather quantitative and qualitative data that support or disprove these hypotheses, 
and measure, share, and learn from the outcomes.

In the case of Recommenders, the evidence is gathered from extensive experience working 
with customers in real life recommendation scenarios. When there is no input from a 
customer, the second source of evidence that is used is popular machine learning
libraries like Tensorflow \cite{abadi2016tensorflow}, PyTorch \cite{paszke2017automatic},
Scikit-learn \cite{pedregosa2011scikit} or LigthGBM \cite{ke2017lightgbm}.

One example to illustrate EBD is the definition of the metrics interface in Python and 
PySpark. The definition in Python as based of functions:

\begin{minted}{python}
    from reco_utils.evaluation.python_evaluation import rmse
    result = rmse(df_test, df_predictions)
\end{minted}

whereas the definition in PySpark is based of classes:
\begin{minted}{python}
    from reco_utils.evaluation.spark_evaluation import SparkRatingEvaluation
    rating_eval = SparkRatingEvaluation(df_test, df_prediction)
    result = rating_eval.rmse()
\end{minted}

The selection of classes in PySpark and functions in Python, instead of having a unified
interface, is derived from the EDB principle. Our customers and machine learning users
that code in Python will naturally choose a function when creating metrics, as it is
defined in Scikit-learn \cite{pedregosa2011scikit}. On the contrary, PySpark users 
will tend to code based on classes as defined in \cite{meng2016mllib}. 

\subsubsection{Test Pipeline}

The test pipeline in Recommenders is slightly more complicated than in most machine learning
libraries \cite{abadi2016tensorflow,paszke2017automatic,pedregosa2011scikit,ke2017lightgbm}.
Apart from standard {\em unit tests} with PyTest \cite{krekel2004pytest} of the utilities, 
the Jupyter notebooks are also tested. To perform these tests we use Papermill 
\cite{nteract2017papermill}, which allows for programmatic execution of a notebook. The
unit tests are executed in every pull request and ensure that the utilities and 
notebooks run without an error.

We have also included nightly tests composed by {\em smoke} and {\em integration tests}
\cite{gonzalez-fierro2018beginners}. In the smoke tests, we run the notebooks with a 
small dataset or a small number of epochs to make sure that, apart from running, they 
provide reasonable metrics. In the following example we show how to make sure that
the precision at k of SAR algorithm with Movielens 100k dataset is tested.

\begin{minted}{python}
    TOL = 0.05
    @pytest.mark.smoke
    def test_sar_single_node_smoke(notebooks):
        notebook_path = notebooks["sar_single_node"]
        pm.execute_notebook(
            notebook_path,
            OUTPUT_NOTEBOOK,
            parameters=dict(TOP_K=10, MOVIELENS_DATA_SIZE="100k"),
        )
        results = pm.read_notebook(OUTPUT_NOTEBOOK).dataframe.set_index("name")["value"]
        assert results["precision"] == pytest.approx(0.326617179, TOL)
\end{minted}

The smoke tests are a small version of the integration tests. While the integration tests
use bigger datasets with more epochs and can take from 30min to several hours, the smoke tests are quicker and
will check the correctness of all the notebooks before executing the integration tests.

At the time of writing we have over 400 tests, between Windows and Linux. Furthermore, we have two group of tests,
one for master branch and one for staging branch, which is our development branch.

The existence of a development branch, staging, is tightly related to the test pipeline. The paradigm is similar to
production and pre-production environments. The production branch is master, while the pre-preproduction branch is staging.
The objective is to make sure that the code in master always work. When making a pull request, we will
do it against staging, instead of master. The nightly tests will execute in staging and make sure that the new code 
works as expected, apart from not failing. When staging is stable, we will make a pull request to master.  


\subsubsection{Other Design Patterns}

We introduced a number of design patters that are standard in the industry and well-known
in the open source community. For instance, \textit{don't repeat yourself}, \textit{single
responsibility} or the Zen of Python\footnote{\url{https://www.python.org/dev/peps/pep-0020/}}. 

The following example illustrates the pattern \textit{explicit is better than implicit}.
An implicit read function would be: 
\begin{minted}{python}
    def read(filename):
        # code for reading a csv or json
        # depending on the file extension
\end{minted}

\begin{minted}{python}
    def read(filename):
        # code for reading a csv or json
        # depending on the file extension
\end{minted}

whereas the explicit example would be:
\begin{minted}{python}
    def read_csv(filename):
        # code for reading a csv

    def read_json(filename):
        # code for reading a json
\end{minted}





\section{System Design}

\input{system}

\section{Structure of the Repository}

Microsoft-Recommenders is based on best practices for building recommendation systems, which have been learned from application with production-ready systems.
As a consequence, we have followed an {\em end-to-end} approach, which is not limited to just the training of an algorithm, but encompasses the following components, typical in a recommendation pipeline:

\begin{itemize}
\item Data preparation: Preparing and loading data for each recommender algorithm
\item Model: Building models using various classical and deep learning recommender algorithms such as Alternating Least Squares (ALS) or eXtreme Deep Factorization Machines (xDeepFM).
\item Evaluate: Evaluating algorithms with offline metrics
\item Model Select and Optimize: Tuning and optimizing hyperparameters for recommender models
\item Operationalize: Operationalizing models in a production environment on Azure
\end{itemize}

Several utilities are provided in \url{https://github.com/microsoft/recommenders/blob/master/reco_utils} to support common tasks such as loading datasets in the format expected by different algorithms, evaluating model outputs, and splitting training/test data. Implementations of several state-of-the-art algorithms are provided for self-study and customization in one's own applications.

\subsection{Data Preparation}

\input{dataprep}

\subsection{Recommendation Algorithms}

\subsection{Recommendation Algorithms}

The table below lists recommender algorithms currently available in the repository. Notebooks are linked under the Environment column when different implementations are available.

% A couple of lines only for each algorithm	mentioning the computational environment and what type (CF, hybrid etc.)

\subsubsection{ALS}


\subsubsection{SAR}

\subsubsection{LightGBM}

\subsubsection{NCF}

\subsubsection{xDeepFM}

\subsubsection{RBM}

\subsubsection{SVD}

\subsubsection{VW}

\subsubsection{DKN}

\subsubsection{FAST}

\subsubsection{RLRMC}

\subsubsection{Wide \& Deep}


\subsection{Offline Evaluation}

\subsection{Offline Evaluation}

In this directory, a notebook is provided to illustrate evaluating models using various performance measures which can be found in reco utils.

Notebook	Description
evaluation	Examples of different rating and ranking metrics in Python+CPU and PySpark environments.
Two approaches for evaluating model performance are demonstrated along with their respective metrics.

Rating Metrics: These are used to evaluate how accurate a recommender is at predicting ratings that users gave to items
Root Mean Square Error (RMSE) - measure of average error in predicted ratings
R Squared (R2) - essentially how much of the total variation is explained by the model
Mean Absolute Error (MAE) - similar to RMSE but uses absolute value instead of squaring and taking the root of the average
Explained Variance - how much of the variance in the data is explained by the model
Ranking Metrics: These are used to evaluate how relevant recommendations are for users
Precision - this measures the proportion of recommended items that are relevant
Recall - this measures the proportion of relevant items that are recommended
Normalized Discounted Cumulative Gain (NDCG) - evaluates how well the predicted items for a user are ranked based on relevance
Mean Average Precision (MAP) - average precision for each user normalized over all users
Arear Under Curver (AUC) - integral area under the receiver operating characteristic curve
Logistic loss (Logloss) - the negative log-likelihood of the true labels given the predictions of a classifier

 

\subsection{Model Selection}

\subsection{Model Selection}

In this directory, notebooks are provided to demonstrate how to tune and optimize hyperparameters of recommender algorithms with the utility functions (reco utils) provided in the repository.

Notebook	Description
%tuning_spark_als	Step by step tutorials on how to fine tune hyperparameters for Spark based recommender model (illustrated by Spark ALS) with Spark native construct and hyperopt package.
%azureml_hyperdrive_wide_and_deep	Quickstart tutorial on utilizing Azure Machine Learning service for hyperparameter tuning of wide-and-deep model.
%azureml_hyperdrive_surprise_svd	Quickstart tutorial on utilizing Azure Machine Learning service for hyperparameter tuning of the matrix factorization method SVD from Surprise library.
%nni_surprise_svd	Quickstart tutorial on utilizing the Neural Network Intelligence toolkit for hyperparameter tuning of the matrix factorization method SVD from Surprise library.


\subsection{Operationalization}

In this directory, a notebook is provided to demonstrate how recommendation systems developed in a heterogeneous environment (e.g., Spark, GPU, etc.) can be operationalized.

Notebook	Description
%als_movie_o16n	End-to-end examples demonstrate how to build, evaluate, and deploy a Spark ALS based movie recommender with Azure services such as Databricks, Cosmos DB, and Kubernetes Services.
%aks_locust_load_test	Load test example for a recommendation system deployed on an AKS cluster
%lightgbm_criteo_o16n	Content-based personalization deployment of a add click prediction scenario


\subsection{Utilities}

\input{utils}


\subsection{Conclusions - Learnings}

\input{learnings}


\subsubsection*{Acknowledgments}


\section*{References}



\end{document}
